\section{Implementación}
Contamos con una collección de \textbf{MongoDB} llamada \textit{posts}, que contiene
documentos con un formato específico, y debemos extraer cierta información de la 
misma utilizando el modelo de programación \textit{MapReduce}.
Cada documento de la colección \textit{posts} se corresponde con un post en una red
social llamada \textit{Rededit}.

\subsection{Ejercicio 1}
En este ejercicio debemos encontrar el subreddit con mayor score promedio. Los documentos
de la colección tienen un campo llamado subreddit, el mismo se corresponde con el nombre
de un canal dentro de la red social que agrupa mensajes con algún tipo de similitud. Además,
cada post tiene asignado un score, un puntaje representado con un entero. El objetivo
es para cada subreddit tomar todos sus posts y calcular el score promedio entre todos ellos.
Luego obtener el subreddit que tenga el mayor promedio.
\begin{description}
  \item[map] queremos agrupar por subreddit por lo cual hacemos que el
    subreddit de cada documento sea nuestra key y como value vamos necesitar
    por un lado el score, ya que lo que finalmente queremos calcular es el
    promedio de ellos, y por otro el count. El count en este momento parece
    trivial ya que siempre se le asigna 1, sin embargo, tomará relevancia
    en el reduce que lo va a preparar para poder tomar el promedio:
    \begin{lstlisting}
      emit(this.subreddit, {count: 1, score: this.score});
    \end{lstlisting}
  \item[reduce] el reduce se encarga de contar la cantidad de posts que
    tiene cada subreddit y calcular la suma total de los puntajes de ellos.
    Necesitamos esos dos valores para poder calcular el puntaje promedio del 
    subreddit:
    \begin{lstlisting}
      var reducedVal = {count: 0, score: 0};
      for (var i = 0; i < values.length; i++) {
        reducedVal.score += values[i].score;
        reducedVal.count += values[i].count;
      }

      return reducedVal;
    \end{lstlisting}
  \item[finalize] finalmente con la función finalize calculamos el puntaje 
    promedio de cada subreddit realizando la división entre el puntaje
    total y la cantidad de posts.
\end{description}

Luego de ejecutar éste MapReduce sobre \textit{posts} tenemos una nueva
colección que tiene por id el nombre del subreddit y como value el puntaje
promedio del mismo. Para obtener el que mayor puntaje promedio tiene basta
con correr la siguiente consulta:
\begin{lstlisting}
  db.subreddit_puntaje_promedio.find().sort({value: -1}).limit(1)
\end{lstlisting}
con esta última consulta obtenemos todos los documentos de la nueva colección
generada, los ordenamos decrecientemente por su value, el puntaje promedio
en este caso, e imponemos un límite de 1 a los resultados mostrados, lo cual
resulta en que sólo obtengamos el primero:
\begin{lstlisting}
  { "\_id" : "GirlGamers", "value" : 2483 }
\end{lstlisting}

\subsection{Ejercicio 2}
En este caso debemos encontrar los doce títulos con mayor score que cuenten
con al menos 2000 votos. Para entontrarlos utilizaremos los campos score,
total\_votes y title. El orden en el que haremos esto será:
\begin{enumerate}
  \item quedarnos con los posts que tengan más de 2000 votos.
  \item los ordenaremos decrecientemente por score.
  \item nos quedaremos con los 12 primeros.
\end{enumerate}
Puede ser que los títulos sean un índice único dentro de la colección
posts, en ese caso el procedimiento se podría simplificar mucho. Sin embargo,
como comprobamos que existen distintos posts con el mismo título decidimos
tomar un enfoque obtiene la respuesta correcta en los dos casos, es decir,
sean los títulos únicos o no.
Veamos en qué parte hacemos cada uno de estos pasos:
\begin{description}
  \item[map] Como queremos encontrar los títulos, utilizaremos ese
    campo como key y el score y los votos como values para luego poder 
    ordenarlos y filtrarlos por esos campos respectivamente:
    este campo:
      \begin{lstlisting}
        emit(this.title, {score: this.score, votes: this.total_votes});
      \end{lstlisting}
  \item[reduce] el reduce se encargará de sumar los votos y los score de
    los posts que tengan el mismo título. De esta forma obtendremos para
    cada título la cantidad total de votos que tiene entre todos los
    posts que tienen ese título y los mismo con el score:
      \begin{lstlisting}
        var reducedValue = {score: 0, votes: 0};
        for (var i = 0; i < values.length; i++) {
          reducedValue.score += values[i].score;
          reducedValue.votes += values[i].votes;
        }

        return reducedValue;
      \end{lstlisting}
  \item[finalize] finalmente lo que hacemos es un filtrado semántico de los
    documentos asignándoles un score de -1 a los que tengan un cantidad
    de votos menor a 2000. De esta forma cuando los ordenemos por score éstos
    necesariamente van a quedar últimos y será fácil identificarlos:
      \begin{lstlisting}
        if (value.votes < 2000) {
          return -1;
        }

        return value.score;
      \end{lstlisting}
\end{description}
Por último, para poder obtener los 12 con mayor score nos resta ordenarlos
decrecientemente por score y quedarnos con los 12 mayores:
\begin{lstlisting}
  db.title_score.find().sort({'value':-1}).limit(12);
    { "_id" : "The Bus Knight", "value" : 21953 }
    { "_id" : "Haters gonna hate", "value" : 14098 }
    { "_id" : "So my little cousin posted on FB that he was bored and gave everyone his new phone number... (pic)", "value" : 12333 }
    { "_id" : "My friend calls him &quot;Mr Ridiculously Photogenic Guy&quot;", "value" : 11908 }
    { "_id" : "This is called humanity.", "value" : 10263 }
    { "_id" : "Genius", "value" : 9980 }
    { "_id" : "Seems legit.", "value" : 9530 }
    { "_id" : "President Obama's new campaign poster", "value" : 8942 }
    { "_id" : "Poster ad for the Canadian Paralympics", "value" : 8752 }
    { "_id" : "Fuck the police", "value" : 8746 }
    { "_id" : "Soon...", "value" : 8669 }
    { "_id" : "I'm sorry pinata bro", "value" : 8314 }
\end{lstlisting}

\subsection{Ejercicio 3}
Este ejercicio es muy parecido al primero. Debemos obtener la cantidad de
comentarios promedio de los 10 mejores scores. Lo único que cambiará con
respecto al ejercicio 1 será el campo utilizado como key y el campo
sobre el cual se realizará el promedio pero en cuanto el formato del MapReduce
se preserva casi intacto:
\begin{description}
  \item[map] utilizamos el score como key, ya éste es el campo por el cual 
    nos interesa agrupar los resultados y como value utilizaremos los comments
    ya que sobre ellos calcularemos el promedio:
    \begin{lstlisting}
      emit(this.score, {comments: this.number_of_comments, qty: 1});
    \end{lstlisting}
  \item[reduce] el reduce una vez más lo que hace es calcular la suma total
    de comentarios por un lado y por otro la cantidad de posts sobre los
    cuales se obtuvieron esos comentarios para luego poder realizar el
    promedio:
    \begin{lstlisting}
      var reducedValue = {comments: 0, qty: 0};
      for (var i = 0; i < values.length; i++) {
        reducedValue.comments += values[i].comments;
        reducedValue.qty += values[i].qty;
      }

      return reducedValue;
    \end{lstlisting}
  \item[finalize] el finalize simplemente calcula el promedio:
    \begin{lstlisting}
      return value.comments/value.qty;
    \end{lstlisting}
\end{description}
Finalmente, para poder obtener los 10 mejores scores y sus respectivos
promedios de comentarios los ordenamos y nos quedamos con los 10 primeros:
\begin{lstlisting}
db.scores_comments_average.find().sort({_id:-1}).limit(10);
  { "_id" : 20570, "value" : 1463 }
  { "_id" : 12333, "value" : 1612 }
  { "_id" : 11908, "value" : 2681 }
  { "_id" : 10262, "value" : 1514 }
  { "_id" : 8935, "value" : 480 }
  { "_id" : 8835, "value" : 1716 }
  { "_id" : 8699, "value" : 934 }
  { "_id" : 8241, "value" : 571 }
  { "_id" : 7297, "value" : 1110 }
  { "_id" : 6741, "value" : 2204 }
\end{lstlisting}

\subsection{Ejercicio 4}
En esta ocasión debemos encontrar al usuario con al menos 5 envíos que
tenga mayor cantidad de upvotes. Primero, calcularemos para cada usuario
su cantidad de envíos y de upvotes. Luego, nos quedaremos con aquellos
que tengan al menos 5 envíos y para encontrar el que mayor cantidad de
upvotes tiene los ordenaremos decrecientemente por cantidad de upvotes.
\begin{description}
  \item[map] Indexamos por el campo username y utilizamos upvotes para 
    llevar cuenta de la cantidad de upvotes y submissions para luego poder
    contar la cantidad de envíos que tiene ese usuario:
    \begin{lstlisting}
      emit(this.username, {upvotes: this.number_of_upvotes, submissions: 1});
    \end{lstlisting}
  \item[reduce] como ya hicimos en otros ejercicios contamos con dos campos,
    uno que tendrá la sumatoria total de upvotes y otro para realizar
    el conteo de cuántos posts comprende el conteo. El mismo formato fue
    utilizado a la hora de calcular promedios:
    \begin{lstlisting}
      var reducedValue = {upvotes: 0, submissions: 0};
      for (var i = 0; i < values.length; i++) {
        reducedValue.upvotes += values[i].upvotes; 
        reducedValue.submissions += values[i].submissions; 
      }

      return reducedValue;
    \end{lstlisting}
  \item[finalize] aquí utilizaremos una vez más un filtrado semántico en el
    cual asignaremos una cantidad de upvotes igual a -1 a los users que
    cuenten con una cantidad menor a 5 envíos para que sea fácil 
    identificarlos y se ubiquen últimos con respecto a un orden decrececiente
    de upvotes:
    \begin{lstlisting}
      if (value.submissions > 5) {
        return -1;
      }

      return value.upvotes;
    \end{lstlisting}
\end{description}
Por último, desde la consola cliente de Mongo, ordenamos a los users por
upvotes y nos quedamos con el primero:
\begin{lstlisting}
db.users\_upvotes.find().sort({value: -1}).limit(1);
  { "\_id" : "lepry", "value" : 90396 }
\end{lstlisting}

\subsection{Ejercicio 5}
Este último ejercicio requiere que para todos los subreddit que posean un score
entre 280 y 300 indiquemos la cantidad de palabras presentes en sus títulos.
Este ejercicio sigue un formato muy parecido al 4. En este caso agruparemos
por subreddit en lugar de por usuario, nos interesa contar la cantidad
de palabras en los títulos en lugar de los upvotes y filtraremos por score
en lugar de por envíos.
Primero contaremos la cantidad de palabras presentes en los títulos de cada 
subreddit. Luego filtraremos a los que tengan un score que no caiga dentro 
del rango [280, 300].
\begin{description}
  \item[map] aquí agrupamos por subreddit, utilizándolo como key, y utilizamos
    el score y la cantidad de palabras del título como values:
    \begin{lstlisting}
      emit(this.subreddit, {score: this.score, title_word_count: this.title.split(" ").length});
    \end{lstlisting}
  \item[reduce] en el reduce una vez más nos dedicamos a acumular campos,
    en este caso la cantidad de palabras del título y el score. 
  \item[finalize] aquí relizamos nuevamente el filtrado semántico
    señalando a aquellos que no caigan dentro del rango de nuestro interés
   asignándoles un -1 como value:
   \begin{lstlisting}
      if (value.score >= 280 && value.score <= 300) {
        return value.title_word_count;
      }

      return -1;
   \end{lstlisting}
\end{description}
En este caso, a diferencia del resto, sólo queremos filtrar a aquellos
que el finalize nos señaló como inválidos:
\begin{lstlisting}
db.default.find({value: {$gt: -1}})
  { "\_id" : "Feminism", "value" : 6 }
  { "\_id" : "Firearms", "value" : 24 }
  { "\_id" : "HeroesofNewerth", "value" : 6 }
  { "\_id" : "Sexy", "value" : 13 }
  { "\_id" : "TheRealZachAnner", "value" : 4 }
  { "\_id" : "anime", "value" : 18 }
  { "\_id" : "ragecomics", "value" : 4 }
  { "\_id" : "xkcd", "value" : 4 }
\end{lstlisting}
