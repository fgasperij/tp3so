\section{Investigación}

\subsection{Problema}
Problema: cuando compartían un mismo cluster jobs de distina duración la performance era mala.

\subsection{Perspectiva}
Solución propuesta: cambiar el scheduler de Hadoop. FAIR scheduler.

\subsection{Obstáculos encontrados}
MapReduce presenta dos problemas al esquema que delinea el FAIR scheduler.

\subsubsection{Data locality}
Perjudica la localidad de los datos de las tasks. Al trabajar en un sistema distribuido
los datos que una task debe procesar no necesariamente se encuentran en el mismo
nodo en el que se le asigno un slot. Lo ideal es que sí para que el tiempo que toma
transmitir la información de donde sea que esté hasta donde la task la necesita
sea lo más rápido posible. En un cluster con 2 niveles de localidad podemos tener
varias formas de localidad:
\begin{enumerate}
  \item node locality: la información se encuentra en el nodo que nos asignaron, es lo mejor.
  \item rack locality: la información se encuentra en un nodo con el que compartimos el 
    rack. La velocidad de transmisión para nodos en un mismo rack es menor a la que se consigue
    en el nodo mismo pero mucho mayor que la que existe entre distintos racks.
\end{enumerate}
Llaman al problema head-of-line scheduling el cual consiste en que los jobs más largos
consiguen una mayor localidad de sus datos ya que al tener más nodos con sus datos la 
probabilidad de que caigan en uno de ellos es mayor. Se llama head-of-line scheduling
porque lo que sucede es que cuando un nodo se desocupa se le envía el job que primero
está en la cola sin tener en consideración cuestiones referidas a la localidad de sus datos.
Sticky slots es otro problema relacionado, que sucede cuando la distribución de slots 
entre los jobs es pareja por lo cual los jobs nunca dejan sus slots originales.
La solucióń propuesta es lo que llamaron delay scheduling. Como vimos, seguir el orden 
estricto de la cola impone la posibilidad de asignar un slot a un job que no cuenta 
con data local. Por lo tanto, el delay scheduling propone que antes de asignar un job a un
nodo se considere la localidad de los datos. Si el primer job no cuenta con localidad suficiente
en el nodo disponible se pasa a considerar otro job y no se le asigna el nodo al primero.
Para prevenir la inanición de estos jobs que no pueden encontrar localidad hay 
dos tiempos máximos que se contemplan: uno de rack y otro de off-rack en el cual sin 
importar si es local o no se le asigna un slot. Es decir, primero se intenta conseguir
un nodo con máxima localidad, superado cierto umbral uno con localidad de rack y superado
el segundo umbral cualquier nodo disponible será asignado al job.

\subsubsection{Interdependencia de Reduce/Map}
Los reduce slots son ocupados por reducers que están esperando que 
sus correspondientes maps terminen y esto puede llegar a tomar mucho tiempo lo cual
redunda en desperdicio del slot. Esto pasa mayormente cuando hay jobs largos. Hadoop
larga las tasks de reduce apenas terminan las primeras map tasks para que puedan empezar a copiar
la información. No se puede resolver el problema simplemente haciendo que las reduce tasks empiecen
más tarde porque por la forma en la que están implementadas en Hadoop inclusive con una sola tasks
puede haber desperdicio ya que una vez que se toma el slot pueden haber intervalos de tiempo 
en los que se dedica a hacer procesamiento intensivo y otros en los que hace IO intensivo, durante
esos intervalos uno o el otro se desperdicia. Lo ideal sería poder superponer las operaciones de IO 
y de procesamiento para que ocurran simultáneamente.
La solución que proponenen se llama copy-compute splitting. La idea es dividir las reduce tasks en dos unidades 
lógicas separadas: copy tasks y compute tasks con diferentes tipos de admisión de control. Las
copy taks se encargarán de llevar y hacer merge de la salida de los maps y las compute tasks
van a ser las encargadas de aplicar las funciones de reduce. Una de las posibles implementaciones
era con procesos separados pero la necesidad de tener memoria compartida y las complejidades 
que eso implicaba provocó que descartaran la idea. Su implementación, llamada Compute Phase
Admission Control (CPAC), consiste en que los nodos tengan dos límites: uno para la cantidad
de reducers que pueden estar simultáneamente computando o copiando datos, éste es igual a 
la cantidad de slots que tienen, y otro, que es varias veces mayor a éste, que es la cantidad
de reducers que pueden estar compitiendo por estos lugares. De esta forma se puede lograr 
que diferentes jobs puedan estar copiando datos y computando simultáneamente. Como el código de
CPAC reusa el código de Hadoop de las reduce tasks no pierde ninguno de sus beneficios.
Un potencial problema de esta implementación como fue descripta es que la memoria que 
utilizan las copy tasks debe ser reducida porque no debe perturbar a las computing tasks,
recordemos que en la fase de cómputo los reducers utilizan mucha memoria.
Compararon la performance de FIFO contra FAIR scheduling con y sin copy-compute splitting.
Con respecto al response time en batch jobs FIFO supera al FAIR scheduling en todas sus formas.
Sin embargo, FAIR supera a FIFO cuando se comparan sus slowdowns, que es cuánto se aleja 
el response time obtenido del que hubiera tenido si hubiera tenido un cluster privado dedicado.
Es ésta comparación la que muestra que FAIR scheduling resulta superior en contexto de multi-usuarios.

\subsection{Discusión}
En todo momento lo que se intenta equilibrar con este trabajo es el aislamiento de cada usuario
y la utilización que se hace de los recursos. Por aislamiento de cada usuario nos referimos 
a algo parecido a lo que mencionamos como la métrica de slowdown, y es que cada usuario pueda
llegar a tener la sensación, en response time, de que tiene un cluster privado, más pequeño,
pero exclusivo. Se partió de un FIFO scheduler que utiliza muy bien los recursos pero no
provee ningún tipo de aislamiento para los usuarios. Lo que lograron con el FAIR scheduler
es ubicarse en un punto más intermedio entre el aislamiento y la utilización óptima de
los recursos. De esta forma, en un cluster donde tienen que convivir jobs muy distintos como
queries interactivas, jobs de producción o batchs, pueden hacerlo de una forma más 
balanceada.

\subsection{Conclusión}
Los clusters compartidos presentan un gran beneficio cuando se comparte información
entre muchos usuarios ya que permite reducir los costos y una mejor utilización
de los recursos. Como el uso que reciben se ha orientado en muchos a casos a ambientes
en los cuales deben responder a múltiples usuarios considerar la implementación
de políticas como el FAIR scheduling son una buena opción.
